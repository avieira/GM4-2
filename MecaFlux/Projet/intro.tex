\addcontentsline{toc}{section}{Introduction}
Dans le cadre de l'option Mécanique des Fluides, nous avions à utiliser le logiciel Fluent dans le cadre d'un projet.\\
Notre projet consiste à étudier l'écoulement du vent autour d'un camion à différentes vitesses. Plus précisément, nous avions quatre configurations pour le camion :
\begin{itemize}
	\item La cabine seule, sans remorque
	\item Le camion avec une remorque de la même hauteur que la cabine
	\item Le camion avec une remorque plus haute que la cabine
	\item Le camion avec une remorque plus basse que la cabine
\end{itemize}
Ces configurations ont été étudiées à deux vitesses : 60 et 110 km.h$^{-1}$. Pour cela, différents maillages ont été fournis, correspondant à chacune des configurations décrites précédemment.

\bigskip
Dans un premier temps, nous analyserons les points suivants :
\begin{itemize}
	\item Les différents maillages (qualité, finesse, vraisemblance à la réalité...)
	\item Les écoulements obtenus dans chaque cas d'étude (plus particulièrement vitesse et pression)
	\item Les efforts sur les parois
\end{itemize}
On présentera tout au long de cette étude la démarche suivie ainsi que les différents réglages effectués, qui seront justifiés.

\bigskip
Enfin, à partir de ces observations, on finira par répondre aux questions suivantes :
\begin{itemize}
	\item Quelles remarques faire à propos de la consommation et de la tenue de route de ce véhicule ? 
	\item Comment peut-on expliquer les différences de résultats entre les cas ?
	\item Comment modifier uniquement la géométrie de la remorque pour transporter plus de marchandises aux mêmes vitesses sans forcément consommer plus ?
	\item Comment modifier la géométrie de la cabine et la géométrie de la remorque pour transporter plus de marchandises aux mêmes vitesses sans forcément consommer plus ?
\end{itemize}

