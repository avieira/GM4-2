\addcontentsline{toc}{section}{Introduction}
Le but de ce projet est de modéliser un site internet de nouvelles en ligne. On retrouve bien souvent des sites internet qui réunissent en leur sein deux services :
\begin{itemize}
	\item Une ou plusieurs pages présentant plusieurs nouvelles (news), qui peuvent être commentées
	\item Un forum où les utilisateurs peuvent échangés publiquement (posts) ou de façon plus privée (messages privés, ou MP)
\end{itemize}

\smallskip
Au c\oe ur de tout cela se trouve donc une base d'utilisateurs, qui forment des groupes :
\begin{itemize}
	\item des utilisateurs communs qui postent des commentaires sur les news, sur le forum ou qui s'envoient des messages privés
	\item des modérateurs pour le forum ou les commentaires
	\item des "newseurs" qui postent les nouvelles sur le site, et qui utilisent parfois certaines sources qu'il faut pouvoir citer et retrouver
\end{itemize}

\smallskip
Le forum aura également sa propre organisation interne, que la base de données devra églament présenter :
\begin{itemize}
	\item les posts publics seront inclus dans des sous-sections, eux-mêmes inclus dans des sections
	\item les messages privés s'organisent par sujet qui connaîtront un certain nombre de destinataires qui échangeront toute une conversation
\end{itemize}

\bigskip
La base de données devra donc pouvoir répondre aux problèmes suivants :
\begin{itemize}
	\item Quelles news ont été postées à une date donnée et qui les a posté ?
	\item Où telle source a été utlisée ? Quand ? Et par qui ?
	\item Qui a écrit un commentaire donné ? Sur quelle news ?
	\item Quelles sont les sous-section de telle section ? Quels sont les sujets compris dans telle sous-section ?
	\item Qui a écrit tel post ? Dans quel sujet ? Quand ?
	\item Quels sont les destinataires d'un sujet de message privé ? Qui a écrit tel message privé ? En rapport avec quel sujet ?
	\item Quels sont les conversations par messages privés ou sections sur le forum auquel un utilisateur a accès ?
	\item A quel groupe appartient un utilisateur donné ? Qui sont les modérateurs du forum ? 
\end{itemize}

\bigskip
Dans un premier temps, le modèle Entités/Associations sera présenté, puis on en déduira le modèle relationnel. Chacun de ces modèles sera commenté et justifié.
