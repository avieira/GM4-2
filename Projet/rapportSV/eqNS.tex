
\section{Un peu d'hydrodynamique}
Les équations de Saint-Venant sont utilisées lorsque la longueur sur laquelle nous travaillons est bien plus grande que sa hauteur. C'est par exemple le cas des océans, des tsunamis. Elles sont utilisées dans de nombreux domaines tels que la métérologie, la modélisation des océans. On peut aussi les utiliser pour modéliser des vagues rebondissants contre le bord dans une baignoire remplie d'eau ou le comportement d'un liquide après que l'on ait lancé une pierre dedans. Par exemple dans le manuel d'utilisation de Phoenics sur les équations de Saint-Venant, elles sont utilisées pour modéliser le comportement d'un flot dans un canal ouvert lorsque celui-ci forme un virage. Cet article est disponible à l'adresse suivante : \url{http://www.cham.co.uk/phoenics/d_polis/d_lecs/shallwat/shallwat.htm#References} \\
\section{Les équations de Navier-Stokes}
Les équations de Navier-Stokes sont en réalité constituées de deux équations :
\begin{itemize}
	\item Une équation de continuité (conservation de la masse)
	\item Équation de quantité de mouvement
\end{itemize}

\bigskip
Dans la suite, nous noterons :
\begin{itemize}
	\item $\overrightarrow{U}(U,V,W)$ le vecteur vitesse de composantes $U, V$ et $W$
	\item $\rho$ la densité
	\item $p$ la pression
	\item $\Omega$ le volume de contrôle
\end{itemize}

\bigskip
On note $M(t)$ la masse de fluide dans le volume $\Omega$ à l'instant $t$. On l'exprime par :
	\begin{equation} \label{II-2} M(t)=\int_{\Omega} \rho d\Omega \end{equation}

	\subsection{Équation de continuité}
L'équation de continuité s'écrit :
	\begin{equation} \label{II-2-1-1} \frac{d M(t)}{dt}=0 \end{equation}

On transforme l'expression de \ref{II-2-1-1} :
\begin{eqnarray*}
	\frac{d M(t)}{dt}&=&\frac{\partial}{\partial t} \int_{\Omega} \rho d\Omega + \underbrace{\int_{\partial\Omega} \rho \overrightarrow{U}.\overrightarrow{n} d\sigma}_{\text{Flux net à travers la surface} \partial\Omega}\\
			&=& \int_{\Omega} \frac{\partial \rho}{\partial t} d\Omega + \int_{\partial \Omega} \rho \overrightarrow{U}.\overrightarrow{n} d\sigma \\
			&=& \int_{\Omega} \left[ \frac{\partial \rho}{\partial t} + div\left(\rho \overrightarrow{U}\right)\right] d\Omega \text{ (Ostrogradski)}\\
			&=& 0
\end{eqnarray*}

Or, on raisonne pour $\Omega$ quelconque, d'où :
	\begin{equation} \label{II-2-1-2} \frac{\partial \rho}{\partial t} + div\left(\rho \overrightarrow{U}\right)=0 \end{equation}

	\subsection{Equation de quantité de mouvement}
L'équation de quantité de mouvement s'écrit :
	\begin{equation} \label{II-2-2-1} \frac{d}{dt} \left( M\overrightarrow{U}\right) =\sum \overrightarrow{F} \end{equation}

On va rééccrire chacun des membres de l'équation de \ref{II-2-2-1}. 
\begin{eqnarray*}
	\frac{d}{dt} \left(M\overrightarrow{U}\right) &=& \frac{d}{dt} \int_{\Omega} \rho \overrightarrow{U} d\Omega \\
							&=& \int_{\Omega} \frac{\partial}{\partial t} \left( \rho \overrightarrow{U} \right) d\Omega + \int_{\partial \Omega} \rho \overrightarrow{U}\left(\overrightarrow{U}.\overrightarrow{n}\right) d\sigma \\
\end{eqnarray*}

D'où :
\begin{equation} \frac{d}{dt} \left(M\overrightarrow{U}\right)= \int_{\Omega} \left[ \frac{\partial}{\partial t} \left( \rho \overrightarrow{U} \right) + div\left( \rho \overrightarrow{U}\otimes \overrightarrow{U}\right)\right] d\Omega \label{II-2-2-2} \end{equation}

On s'occupe à présent de l'autre membre : que vaut le bilan des forces ? On prend comme hypothèse que le bilan des forces est égal à la somme d'une force massique $\overrightarrow{F_M}$ et d'une force surfacique $\overrightarrow{F_S}$. Chacune de ces forces s'expriment ainsi :
\begin{equation} \label{II-2-2-3} \overrightarrow{F_M}=\int_{\Omega} \rho\underbrace{\left( g-2\overrightarrow{\Omega}\wedge \overrightarrow{U}\right)}_{=f} d\Omega \end{equation}
avec :
\begin{itemize}
	\item $\rho$ la densité
	\item $g$ la force de pesanteur
	\item $\overrightarrow{\Omega}$ vecteur de rotation terrestre
	\item $f$ la résultante des forces massiques
\end{itemize}

\begin{equation} \label{II-2-2-4} \overrightarrow{F_S}=\int_{\Omega} div\left( -pI+\underbrace{2\rho\nu D}_{=\tau}\right) d\Omega \end{equation}
avec :
\begin{itemize}
	\item $p$ la pression
	\item $I$ la matrice identité
	\item $\rho$ la densité
	\item $\nu$ coefficient de viscosité cinématique
	\item $D$ le tenseur des taux de déformation
	\item $\tau$ le tenseur des contraintes visqueuses
\end{itemize}

\bigskip
On tire donc de \ref{II-2-2-3} et \ref{II-2-2-4} :
\begin{equation}\label{II-2-2-5} \sum \overrightarrow{F}= \int_{\Omega} \left( \rho f + div\left( -pI+\tau\right) \right)d\Omega\end{equation}

Or, on a égalité pour tout $\Omega$ entre \ref{II-2-2-2} et \ref{II-2-2-5}, d'où :
\begin{equation} \label{II-2-2-6} \frac{\partial}{\partial t} \left( \rho \overrightarrow{U} \right) + div\left( \rho \overrightarrow{U} \otimes \overrightarrow{U}\right) = \rho f -\nabla p + div(\tau) \end{equation}

	\subsection{Système d'équations}
Les équations de Navier-Stokes se résument donc par :
\[\left\{ \begin{array}{c c c}
	\frac{\partial \rho}{\partial t} + div\left(\rho \overrightarrow{U}\right)&=&0\\
	\frac{\partial}{\partial t} \left( \rho \overrightarrow{U} \right) + div\left( \rho \overrightarrow{U} \otimes \overrightarrow{U}\right) &=& \rho f -\nabla p + div(\tau)
\end{array}\right.\]

En introduisant le vecteur vitesse dans le repère cartésien $\overrightarrow{U}=\begin{pmatrix} U \\ V \\ W \end{pmatrix}$ et $f=(fV,-fU,f_z)^T$, on arrive au système d'équations suivant :
\[\left\{ \begin{array}{c c c}
	\frac{\partial \rho}{\partial t} + \frac{\partial U}{\partial x} + \frac{\partial V}{\partial y} +\frac{\partial W}{\partial z}&=&0\\
     \frac{\partial U}{\partial t} + \frac{\partial U^2}{\partial x} + \frac{\partial (UV)}{\partial y} +\frac{\partial (UW)}{\partial z} &=& -\frac{1}{\rho}\frac{\partial p}{\partial x} + \Delta(\nu U)+fV\\
	\frac{\partial V}{\partial t} + \frac{\partial (UV)}{\partial x} + \frac{\partial V^2}{\partial y} +\frac{\partial (VW)}{\partial z} &=& -\frac{1}{\rho}\frac{\partial p}{\partial y} + \Delta(\nu V)-fU\\
	\frac{\partial W}{\partial t} + \frac{\partial (UW)}{\partial x} + \frac{\partial (VW)}{\partial y} +\frac{\partial W^2}{\partial z} &=& -\frac{1}{\rho}\frac{\partial p}{\partial z} + \Delta(\nu W)+f_z
\end{array}\right.\]


\section{Moyenne des équations de Navier-Stokes}
Afin d'obtenir les équations de Saint-Venant, nous commençons par faire quelques hypothèses :
	\paragraph{Hypothèses de Boussinesq}
On considère ici la densité $\rho$ comme constante par rapport au temps et à l'espace. Cela permet de simplifier le système d'équations de Navier-Stokes, les termes tels que $\frac{\partial \rho}{\partial t}$ devenant nuls.

	\paragraph{Hypothèse de pression hydrostatique}
On considère également que l'acceleration du mouvement sur la verticale est négligeable devant l'accélération due à la gravité. Cela permet de négliger les termes $\frac{\partial W}{\partial x}$ et $\frac{\partial W}{\partial y}$. Cela débouche, dans les équations de quantité de mouvement, à l'égalité :

	\[\frac{\partial p}{\partial z}=-\rho g\]

	\paragraph{Vitesse moyenne}
On considère $u$ la vitesse moyenne suivant $x$ sur la verticale et $v$ la vitesse moyenne suivant $y$ sur la verticale :
	\[u=\frac{1}{h} \int_{Z_f}^{Z_s} U dz\]
	\[v=\frac{1}{h} \int_{Z_f}^{Z_s} V dz\]
avec $Z_s$ la hauteur de la surface et $Z_f$ la hauteur du fond, et $h=Z_s-Z_f$.

	\paragraph{Transformation de l'équation de continuité}
On va chercher à réexprimer l'équation de continuité $div(\overrightarrow{U})=0$ grâce à ces variables. On utilise pour cela la formule de Leibniz :
\begin{equation} \label{II-3-1}\int_{Z_f}^{Z_s} div(\overrightarrow{U}) dz = \frac{\partial}{\partial x} \int_{Z_f}^{Z_s} Udz + \frac{\partial}{\partial y} \int_{Z_f}^{Z_s} V dz + W(Z_s)-W(Z_f) + U(x,y,Z_f)\frac{\partial Z_f}{\partial x}\end{equation}
\[- U(x,y,Z_s)\frac{\partial Z_s}{\partial x} + V(x,y,Z_f)\frac{\partial Z_f}{\partial y} - V(x,y,Z_f)\frac{\partial Z_s}{\partial y}\]

On fait à présent des hypothèses d'imperméabilité de la surface libre et du fond, c'est-à-dire qu'il n'y a aucun transfert de masse à travers le fond et la surface. Ces hypothèses s'expriment ainsi :
\[U(x,y,Z_s)\frac{\partial Z_s}{\partial x} + V(x,y,Z_s)\frac{\partial Z_s}{\partial y}=W(Z_s)-\frac{\partial Z_s}{\partial t}\]
\[U(x,y,Z_f)\frac{\partial Z_f}{\partial x} + V(x,y,Z_f)\frac{\partial Z_f}{\partial y}=W(Z_f)-\frac{\partial Z_f}{\partial t}\]

L'équation \ref{II-3-1} devient donc :
\[\frac{\partial}{\partial x} \int_{Z_f}^{Z_s} U dz + \frac{\partial}{\partial y} \int_{Z_f}^{Z_s} V dz + \frac{\partial}{\partial t} (Z_s-Z_f)=0\]
\begin{equation}\label{II-3-2} div(h\overrightarrow{u}) + \frac{\partial h}{\partial t} = 0 \end{equation}
avec $\overrightarrow{u}=(u,v)^T$

\paragraph{Equation de quantité de mouvement : membre de gauche}
On va maintenant tenir le même raisonnement sur les équations de quantité de mouvement. On s'intéressera principalement à la première des deux équations, la deuxième étant faite de façon analogue.\\
On repart de l'équation :
\[\frac{\partial U}{\partial t} + \frac{\partial U^2}{\partial x} + \frac{\partial (UV)}{\partial y} +\frac{\partial (UW)}{\partial z} = -\frac{1}{\rho}\frac{\partial p}{\partial x} + \Delta(\nu U)+f_x\]

On s'intéresse uniquement au membre de droite, et on en calcule l'intégrale sur la verticale. On utilise la formule de Leibniz dans la plupart de ces égalités :
\[\int_{Z_f}^{Z_s} \frac{\partial U}{\partial t} dz = \frac{\partial}{\partial t} (hu) + U(x,y,Z_f)\frac{\partial Z_f}{\partial t} - U(x,y,Z_s) \frac{\partial Z_s}{\partial t}\]
\[\int_{Z_f}^{Z_s} \frac{\partial UW}{\partial z} dz =U(x,y,Z_s)W(x,y,Z_s)-U(x,y,Z_f)W(x,y,Z_f)\]
\[\int_{Z_f}^{Z_s} \frac{\partial U^2}{\partial x} dz = \frac{\partial}{\partial x} \int_{Z_f}^{Z_s} U^2 dz + U^2(x,y,Z_f)\frac{\partial Z_f}{\partial x} - U^2(x,y,Z_s) \frac{\partial Z_s}{\partial x}\]
\[\int_{Z_f}^{Z_s} \frac{\partial UV}{\partial y} dz = \frac{\partial}{\partial y} \int_{Z_f}^{Z_s} UV dz + U(x,y,Z_f)V(x,y,Z_f)\frac{\partial Z_f}{\partial y} - U(x,y,Z_s)V(x,y,Z_s) \frac{\partial Z_s}{\partial y}\]

\bigskip
Dans les deux dernières équations, on retrouve des termes qu'on peut encore simplifier :
\begin{eqnarray*}
	\frac{\partial}{\partial y} \int_{Z_f}^{Z_s} UV dz &=& \frac{\partial}{\partial y} \int_{Z_f}^{Z_s} (u+U-u)(v+V-v) dz \\
							&=&\frac{\partial}{\partial y} (huv) + \frac{\partial}{\partial y} \int_{Z_f}^{Z_s} (U-u)(V-v) dz + \frac{\partial}{\partial y} \int_{Z_f}^{Z_s} (uV+vU-2uv) dz \\
							&=&\frac{\partial}{\partial y} (huv) + \frac{\partial}{\partial y} \int_{Z_f}^{Z_s} (U-u)(V-v) dz + \frac{\partial}{\partial y} \underbrace{\left[ uhv + vhu - 2uvh \right]}_{=0} \\
							&=& \frac{\partial}{\partial y} (huv) + \frac{\partial}{\partial y} \int_{Z_f}^{Z_s} (U-u)(V-v) dz
\end{eqnarray*}

De même, on va avoir :
\[\frac{\partial}{\partial x} \int_{Z_f}^{Z_s} U^2 dz= \frac{\partial}{\partial t} (hu^2) + \frac{\partial}{\partial x} \int_{Z_f}^{Z_s} (U-u)^2 dz\]

Après simplification, on obtient le résultat suivant :
\begin{equation} \label{II-3-3} 
	\int_{Z_f}^{Z_s} \left(\frac{\partial U}{\partial t} + \frac{\partial U^2}{\partial x} + \frac{\partial (UV)}{\partial y} +\frac{\partial (UW)}{\partial z}\right) dz = 
\end{equation}
	\[\frac{\partial}{\partial t} (hu) + \frac{\partial}{\partial x}(hu^2)+\frac{\partial}{\partial y}(huv) + \frac{\partial}{\partial x} \int_{Z_f}^{Z_s} (U-u)^2 dz + \frac{\partial}{\partial y} \int_{Z_f}^{Z_s} (U-u)(V-v) dz\]

On peut démontrer les deux égalités suivantes :
\[\int_{Z_f}^{Z_s} (U-u)^2 dz=\nu_d\frac{\partial}{\partial x}(hu)\]
\[\int_{Z_f}^{Z_s} (U-u)(V-v) dz=\nu_d \frac{\partial}{\partial y} (hu)\]
avec $\nu_d$ le coefficient de dispersion.\\
L'équation \ref{II-3-3} devient ainsi :
\begin{equation} \label{II-3-4}
	\int_{Z_f}^{Z_s} \left(\frac{\partial U}{\partial t} + \frac{\partial U^2}{\partial x} + \frac{\partial (UV)}{\partial y} +\frac{\partial (UW)}{\partial z}\right) dz =\frac{\partial}{\partial t} (hu) + \frac{\partial}{\partial x}(hu^2)
\end{equation}
\[+\frac{\partial}{\partial y}(huv)+\Delta(\nu_dhu)\]

\paragraph{Equation de quantité de mouvement : membre de droite}
On s'intéresse maintenant au membre de gauche. On commence par le terme de pression. En réutilisant l'hypothèse de pression hydrostatique énoncé précédement :
	\[\frac{\partial p}{\partial z}=-\rho g\]
On en déduit que :
	\[p(x,y,z)=-\rho g(Z_s-z)\]
Ainsi, on obtient avec la première équation de quantité de mouvement :
\begin{eqnarray*}
	\int_{Z_f}^{Z_s} -\frac{1}{\rho} \frac{\partial}{\partial x} (\rho g(Z_s-z)) dz &=& \int_{Z_f}^{Z_s} \left[ \underbrace{\frac{\partial}{\partial x} (gz)}_{=0}-\frac{\partial gZ_s}{\partial x} \right] dz \\
4							       &=& -g \frac{\partial Z_s}{\partial x} \int_{Z_f}^{Z_s} dz \\
\end{eqnarray*}
D'où :
\begin{equation} \label{II-2-5} \int_{Z_f}^{Z_s} -\frac{1}{\rho} \frac{\partial}{\partial x} (\rho g(Z_s-z)) dz= -hg\frac{\partial Z_s}{\partial x} \end{equation}

De même, on obtient pour la deuxième équation :
\begin{equation} \label{II-2-6} \int_{Z_f}^{Z_s} -\frac{1}{\rho} \frac{\partial}{\partial y} (\rho g(Z_s-z)) dz= -hg\frac{\partial Z_s}{\partial y} \end{equation}

Pour les termes de Coriolis, l'intégration se fait très facilement :
\begin{equation} \label{II-2-7} \int_{Z_f}^{Z_s} fV dz = fhv \end{equation}
\begin{equation} \label{II-2-8} \int_{Z_f}^{Z_s} fU dz = fhu \end{equation}

\bigskip
Il reste maintenant le terme utilisant le lagrangien. Une fois encore, on utilise la formule de Leibniz :
\[\int_{Z_f}^{Z_s} \nu \Delta(U) dz = div\left( \int_{Z_f}^{Z_s} \nu \nabla U dz \right) - \nu \nabla U . \nabla Z_s + \nu \nabla U . \nabla Z_f\]

Les deux derniers termes de cette équation représentent les contraintes de cisaillement au fond et à la surface dû au vent. Ces forces sont en fait superficielles, mais apparaissent dans les équations de Saint-Venant comme des termes sources appliquées à toute la masse de l'eau, puisque les équations représentent une moyenne sur la verticale. On pose ces deux contraintes comme étant $-\frac{1}{\rho h}\overrightarrow{\tau_f}$ pour le fond et $\frac{1}{\rho}\overrightarrow{\tau_s}$ pour la surface. \\
On s'intéresse plus spécialement au terme $\tau_f$. On peut l'exprimer sous la forme :
	\[\overrightarrow{\tau_f}=\begin{pmatrix} \tau_{fx} \\ \tau_{fy} \end{pmatrix} = \frac{1}{2} \rho C_f \sqrt{u^2+v^2} \overrightarrow{u}\]
$C_f$ étant un coefficient de frottement, souvement déterminé par une des formules suivantes :
\begin{itemize}
	\item la formule de Chézy \[C_f = \frac{2g}{C_h^2}\]
	\item la formule de Manning \[C_f = \frac{2gn^2}{R_h^{1/3}}\]
\end{itemize}

où $C_h$ est le coefficient de Chézy et $R_h$ le rayon hydrolique. 

\bigskip
En conclusion, on arrive à l'équation suivante :
\begin{equation} \label{II-2-9}
	\int_{Z_f}^{Z_s} \nu \Delta(U) dz = div(\nu \nabla(hu)) + \frac{\tau_{sx}}{\rho} - \frac{\tau_{fx}}{\rho}
\end{equation}

\paragraph{Conclusion}
En combinant les équations \ref{II-3-4},  \ref{II-2-5}, \ref{II-2-7} et \ref{II-2-9}, on arrive à l'équation suivante :
\[\frac{\partial}{\partial t}(hu) + \frac{\partial}{\partial x} (hu^2) + \frac{\partial}{\partial y}(huv) = -hg\frac{\partial Z_s}{\partial x} + fhv + div((\nu-\nu_d)(\nabla(hu))) + \frac{\tau_{sx}}{\rho} - \frac{\tau_{fx}}{\rho}\]

En posant $A_H=\nu-\nu_d$, appelé coefficient de diffusion, on obtient :
\begin{equation} \label{II-2-10}
	\frac{\partial}{\partial t}(hu) + \frac{\partial}{\partial x} (hu^2) + \frac{\partial}{\partial y}(huv) = -hg\frac{\partial Z_s}{\partial x} + fhv + A_H\Delta(hu) + \frac{\tau_{sx}}{\rho} - \frac{\tau_{fx}}{\rho}
\end{equation}

De la même façon, l'équation de mouvement dans la direction $y$ nous donne :
\begin{equation} \label{II-2-11}
	\frac{\partial}{\partial t}(hv) + \frac{\partial}{\partial x} (huv) + \frac{\partial}{\partial y}(hv^2) = -hg\frac{\partial Z_s}{\partial y} - fhu + A_H\Delta(hv) + \frac{\tau_{sy}}{\rho} - \frac{\tau_{fy}}{\rho}
\end{equation}

\section{Les équations de Saint-Venant}
\paragraph{Système d'équations}
Les considérations de la sous-section précédente nous donne :
\[\left\{ \begin{array}{c c c}
	\frac{\partial}{\partial x} (hu) + \frac{\partial}{\partial y} (hv) + \frac{\partial h}{\partial t} &=& 0 \\
      \frac{\partial}{\partial t}(hu) + \frac{\partial}{\partial x} (hu^2) + \frac{\partial}{\partial y}(huv) &=& -hg\frac{\partial Z_s}{\partial x} + fhv + A_H\Delta(hu) + \frac{\tau_{sx}}{\rho} - \frac{\tau_{fx}}{\rho}\\
	     \frac{\partial}{\partial t}(hv) + \frac{\partial}{\partial x} (huv) + \frac{\partial}{\partial y}(hv^2) &=& -hg\frac{\partial Z_s}{\partial y} - fhu + A_H\Delta(hv) + \frac{\tau_{sy}}{\rho} - \frac{\tau_{fy}}{\rho}
\end{array}\right.\]

Ces équations peuvent par ailleurs encore se simplifier. Pour cela, on va utiliser la première équation et l'intégrer dans les deux suivantes.

\bigskip
Tout d'abord, remarquons l'égalité suivante :
\begin{equation} \label{II-2-4-1}
	\frac{\partial h}{\partial t} = \frac{\partial Z_s}{\partial t} - \underbrace{\frac{\partial Z_f}{\partial t}}_{=0} = \frac{\partial Z_s}{\partial t}
\end{equation}

Ensuite, intéressons nous aux membres de gauche des équations de quantité de mouvement. On va ici séparer les produits dans les dérivées :
\begin{eqnarray*}
	\frac{\partial}{\partial t}(hu) + \frac{\partial}{\partial x} (hu^2) + \frac{\partial}{\partial y}(huv) &=& h\frac{\partial u}{\partial t} + u\frac{\partial h}{\partial t} + hu\frac{\partial u}{\partial x} + u\frac{\partial hu}{\partial x} + u\frac{\partial hv}{\partial y} + hv\frac{\partial u}{\partial y} \\
														&=& h\left(\frac{\partial u}{\partial t} + u\frac{\partial u}{\partial x} + v\frac{\partial u}{\partial y} \right) + u\underbrace{\left(\frac{\partial h}{\partial t} + \frac{\partial hu}{\partial x} + \frac{\partial hv}{\partial y} \right)}_{(\star)}
\end{eqnarray*}

Or, d'après l'équation de conservation de la masse, on a $(\star)=0$, d'où :
\[\frac{\partial}{\partial t}(hu) + \frac{\partial}{\partial x} (hu^2) + \frac{\partial}{\partial y}(huv) = h\left(\frac{\partial u}{\partial t} + u\frac{\partial u}{\partial x} + v\frac{\partial u}{\partial y} \right)\]

On démontre de la même façon : 
\[\frac{\partial}{\partial t}(hv) + \frac{\partial}{\partial x} (huv) + \frac{\partial}{\partial y}(hv^2)=h\left( \frac{\partial v}{\partial t} + u\frac{\partial v}{\partial x} + v\frac{\partial v}{\partial y}\right)\]

On arrive donc à la formulation suivante :
\begin{equation} \label{II-2-4-3}
	\left\{ \begin{array}{c c c}
	\frac{\partial}{\partial x} (hu) + \frac{\partial}{\partial y} (hv) + \frac{\partial h}{\partial t} &=& 0 \\
      \frac{\partial u}{\partial t} + u\frac{\partial u}{\partial x} + v\frac{\partial u}{\partial y} &=& -g\frac{\partial Z_s}{\partial x} + F_x\\
	 \frac{\partial v}{\partial t} + u\frac{\partial v}{\partial x} + v\frac{\partial v}{\partial y}&=& -g\frac{\partial Z_s}{\partial y} + F_y
	\end{array}\right.
\end{equation}

où $F_x$ et $F_y$ désigneront toutes les forces extérieures normalisées par la hauteur. \\

\subsection{Les équations linéarisées}

On considère que $u$ et $v$ sont petits, et que la hauteur $Z_s$ vaut $Z_s={Z_s}_0+\eta$ où ${Z_s}_0$ serait une hauteur moyenne et $\eta$ une fluctuation autour de la moyenne supposée petite. De même, on néglige les forces extérieures. On pose également $h_0 = {Z_s}_0 - Z_f$

Sous ces hypothèses, les deux équations de quantité de mouvement se simplifient :
\[\left\{\begin{array}{c c c}
	\frac{\partial u}{\partial t} &=& -g\frac{\partial \eta}{\partial x} \\
	\frac{\partial v}{\partial t} &=& -g\frac{\partial \eta}{\partial y} 
\end{array}\right.\]

Pour l'équation de conservation de la masse, on sépare les produits dans les dérivées :
\begin{eqnarray*}
	\frac{\partial h}{\partial t} + \frac{\partial (hu)}{\partial x} + \frac{\partial (hv)}{\partial y} &=& \underbrace{\frac{\partial {Z_s}_0}{\partial t} - \frac{\partial Z_f}{\partial t}}_{=0} + \frac{\partial \eta}{\partial t} + h\frac{\partial u}{\partial x} + \underbrace{u \frac{\partial h}{\partial x} + v\frac{\partial h}{\partial y}}_{\text{négligés}} + h\frac{\partial v}{\partial y} \\
	     &=&\frac{\partial \eta}{\partial t} + ({Z_s}_0 + \underbrace{\eta}_{\text{négligé}} - Z_f)\left(\frac{\partial u}{\partial x} + \frac{\partial v}{\partial y} \right) \\
	     &=&\frac{\partial \eta}{\partial t} + h_0\left(\frac{\partial u}{\partial x} + \frac{\partial v}{\partial y} \right) \\
		&=& 0
\end{eqnarray*}

D'où le système d'équations de Saint-Venant linéarisé :
\begin{equation}\label{II-4-4}
	\left\{ 
	\begin{array}{c c c}
		\frac{\partial u}{\partial t}	& = & -g \frac{\partial \eta}{\partial x}\\
		\frac{\partial v}{\partial t} & = & -g  \frac{\partial \eta}{\partial y}\\
      \frac{\partial \eta}{\partial t} & = & - h_0  (\frac{\partial u}{\partial x}+\frac{\partial v}{\partial y} )\\
		
		
	\end{array}
	\right.
\end{equation}
