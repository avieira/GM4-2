\documentclass[handout]{beamer}

\usepackage[frenchb]{babel}
\usepackage[T1]{fontenc}
\usepackage[utf8x]{inputenc}
\usepackage{minted}
 
\usetheme{Berkeley}
\usecolortheme{crane}
\useinnertheme{rounded}

\title[Saint-Venant]{Présentation projet : les équations de Saint-Venant et la méthode des éléments finis}
\author{Gabrielle \bsc{Collette}, Conrad \bsc{Hillairet} \& Alexandre \bsc{Vieira}}
\institute{INSA de Rouen}
\date{30 mai 2012}


\AtBeginSection[]
{
	\begin{frame}
		\frametitle{Sommaire}
		\tableofcontents[currentsection, hideothersubsections]
	\end{frame}
}

\begin{document}

\begin{frame}
\titlepage
\end{frame}

\begin{frame}
	\frametitle{Sommaire}
	\tableofcontents
\end{frame}

\section{Les équations de Saint-Venant}
\subsection[Hydrodynam.]{Un peu d'hydrodynamique} %Exemples
 
\begin{frame}
	\frametitle{Bouah}

\end{frame}

\subsection[Équations]{Présentation des équations}

\begin{frame}
	\frametitle{Démonstration : grandes idées}

\end{frame}

\begin{frame}
	\frametitle{Équations de Saint-Venant complètes}

\end{frame}

\begin{frame}
	\frametitle{Équations de Saint-Venant linéarisées}

\end{frame}

\section{Méthode des éléments finis}
\subsection[Présentation]{Présentation rapide de la méthode}
\begin{frame}
	\frametitle{Bouah}

\end{frame}

\subsection[Simulation]{Simulation sur un exemple}
\begin{frame}
	\frametitle{Bouah}

\end{frame}

\section[FreeFem++]{Saint-Venant avec FreeFem++}
\subsection[Volumes finis]{La méthode des volumes finis}
\begin{frame}
	\frametitle{Présentation de la méthode}

\end{frame}

\subsection[Simulation FF++]{Simulations avec FreeFem++}
\begin{frame}
	\frametitle{Résultats}

\end{frame}

\section*{Conclusion}
\begin{frame}
	\frametitle{Conclusion}
\end{frame}
\end{document}
